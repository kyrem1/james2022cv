\documentclass[11pt,letterpaper,sans]{moderncv}

\moderncvstyle{classic}
\moderncvcolor{purple}

\usepackage[scale=0.75]{geometry}
\addtolength{\voffset}{-3.5em}
\addtolength{\textheight}{7em}

\quote{``Find what you love and let it kill you''
--- Charles Bukowski}

\name{James}{Harbour}
\title{Curriculum Vitae}
\phone{+1~(205)~876~4085}
\homepage{jameshharbour.net}
\email{james.h.harbour@gmail.com}
\social[github]{kyrem1}
% \lfoot{\emph{Last updated \today}}

\sethintscolumnlength{1cm}


\begin{document}

\makecvtitle % Print the CV title
\vspace*{-1.5em}


\begin{center}
  I am a rising junior at the University of Virginia with a comprehensive undergraduate and graduate-level mathematics background. My primary interests are operator algebras and noncommutative geometry.
\end{center}



% \cventry{Grade}{CourseID}{CourseName}{Semester}{Professor}%
%    {Description \\ \emph{Textbook: }TextAuthor \emph{TextName}.}

\section{Selected Talks}

\cventry{4/23}{Maximal rigidity for $L^2$-cohomology of Groups and Beyond}{UVA Operator Theory seminar}{speaker}{}{} 
\cventry{11/22}{Index Rigidity for type-$ II_1 $ Subfactors}{UVA Operator Theory seminar}{speaker}{}{}
\cventry{11/22}{Construction and examples of the von Neumann dimension of Hilbert modules}{UVA Operator Theory seminar}{speaker}{}{}


\section{Past/Upcoming Conferences}
\cventry{08/23}{Noncommutative Harmonic Analysis and Rigidity Theory in Operator Algebras}{TU Delft, Delft, The Netherlands.}{attendee}{}{}
\cventry{05/23}{Noncommutative Geometry and Operator Algebras (NCGOA) Spring Institute}{Vanderbilt University}{attendee}{}{}
\cventry{01/23}{Joint Math Meetings}{Boston}{attendee}{supported by NSF grant DMS-2035183}{}
\cventry{10/22}{East Coast Operator Algebras Symposium}{Michigan State University}{attendee}{supported by NSF grant DMS-2035183}{}
\cventry{06/22}{Thematic Program in p-adic L-functions and Eigenvarieties - Undergraduate Workshop}{University of Notre Dame}{participant}{supported by NSF grant DMS-1904501}{}
  

\section{University of Virginia (Fall 2021 -- present)}

\cventry{A+}{MATH 8851}{Combinatorial and Geometric Group Theory}{Spring 2023}{Mikhail Ershov}%
   {Free groups and their automorphisms, Amalgamated free products, HNN extensions, Bass-Serre theory, Word growth of groups, Hyperbolic groups, Amenable groups, Burnside type problems, Linear groups. \\ \emph{Textbook: }N/A.}

\cventry{A+}{MATH 8410}{Harmonic Analysis}{Spring 2023}{Gennady Uraltsev}%
   {Distribution theory, interpolation theory, basics of singular integral operators, applications to stochastic differential equations, harmonic analysis on Lie groups.\\ \emph{Textbook: }N/A.}


\cventry{\emph{A+}}{MATH 7410}{Functional Analysis}{Fall 2022}{Benjamin Hayes}%
  {Graduate functional analysis course. Topics include: Banach Space theory: Hahn Banach, Open Mapping theorem, Closed Graph theorem, Principle of Uniform Boundedness. Locally Convex Spaces: Minkowski theory, Hahn-Banach Separation. Weak topologies, Banach-Alaoglu theorem. Extreme points and Krein-Milman theorem. Applications to amenable groups, fixed point theorems. Distribution theory. Applications to Hodge theory.\\ \emph{Textbook: }Conway's \emph{Functional Analysis}}

\cventry{\emph{A}}{MATH 8620}{Algebraic Geometry}{Fall 2022}{Andrei Rapinchuk}%
  {Graduate algebraic geometry course course introducing modern algebraic geometry via the technology of sheaves and schemes.  \\ \emph{Textbook: }Hartshorne's \emph{Algebraic Geometry}}

\cventry{\emph{B+}}{MATH 7820}{Differential Topology}{Fall 2022}{Slava Krushkal}%
  {Graduate differential topology course. Topics covered include manifolds, tangent spaces, Sard's theorem, submersions/immersions, Whitney embedding theorem, orientations, degrees of smooth maps, vector bundles, Riemannian metrics, differential forms, integration of forms on oriented manifolds. \\ \emph{Textbook: }Tu's \emph{Introduction to Manifolds}}

\cventry{\emph{A}}{MATH 7752}{Algebra II}{Spring 2022}{Evangelia Gazaki}%
  {Graduate algebra course covering tensor products and algebras, the classification of finitely generated modules over PIDs, field theory, Galois theory, and infinite Galois theory.  \\ \emph{Textbook: }Dummit and Foote}

\cventry{\emph{A}}{MATH 7800}{Algebraic Topology I}{Spring 2022}{Nicholas Kuhn}%
  {Graduate algebraic topology course covering homology and homotopy theory.  \\ \emph{Textbook: }Hatcher, Vick}

\cventry{\emph{A-}}{MATH 7310}{Real Analysis and Linear Spaces}{Spring 2022}{Benjamin Hayes}%
  {Graduate measure theory course covering abstract measure spaces, integration, modes of convergence, Banach and Hilbert space techniques, $L^p$-spaces, and Fourier analysis. \\ \emph{Textbook: }Folland's \emph{Real Analysis} (1-3,5,6,8).}

\cventry{A+}{MATH 7751}{Algebra I}{Fall 2021}{Andrei Rapinchuk}%
  {Graduate algebra course covering ring theory (Localization, PIDs, UFDs, Euclidean Domains, Group Rings) and module theory (Chain Conditions, Hilbert Basis Theorem, Projective/Injective Modules) with a focus on categorical perspective.   \\ \emph{Textbook: }Dummit and Foote}

\cventry{A}{MATH 7340}{Complex Analysis}{Fall 2021}{Benjamin Hayes}%
  {Graduate complex analysis course. Topics covered include Cauchy’s theorem, Cauchy integral formula, maximum modulus principle, harmonic functions, meromorphic functions, residue theory, Rouche's theorem, normal families of analytic functions, Riemann mapping theorem, Schwartz's lemma, and Schwartz's reflection principle. \\ \emph{Textbook: }Conway's \emph{Functions of One Complex Variable}.}

\cventry{A}{MATH 5657}{Bilinear Forms and Representation Theory}{Fall 2021}{Mikhail Ershov}%
  {First half of the course focused on bilinear forms. Topics in this section included symmetric, skew-symmetric, and alternating bilinear forms (with corresponding diagonalization theorems), sesquilinear and Hermitian forms, self-adjoint and unitary operators, and tensor products (universality, tensor-hom adjunction). The second half of the course covered representation theory. Topics covered included Schur's lemma, unitarizable representations, characters, orthogonality relations, group algebras, regular representation, permutation representations, and a full proof of Burnside's pq theorem. \\ \emph{Textbook: }Steinberg's \emph{Representation Theory of Finite Groups: An Introductory Approach}.}

%\vspace{1cm}

\section{University of South Florida (2019 -- 2021, concurrent dual enrollment during high school)}


\subsection{Mathematics --- Algebra}

\cventry{A+}{MAS 6312}{Algebra II}{Spring 2021}{Brian Curtin}%
  {Graduate algebra course covering basic ring theory (PIDs, UFDs, etc.), field theory (general field extensions, algebraic extensions, splitting fields, Artin's theorem, cyclotomic fields, radical extensions, geometric constructions), Galois theory (Galois extensions, primitive element theorem, Galois connections, Galois groups, insolubility of the quintic), and some commutative algebra (primary and radical ideals, Lasker-Noether theorem, Hilbert's basis theorem, Hilbert's Nullstellensatz).  \\ \emph{Textbook: }Isaacs' \emph{Algebra, a Graduate Course} (16-22, 26-28, 30).}

\cventry{A}{MAS 5301}{Algebra I}{Fall 2020}{Brian Curtin}%
  {Graduate Algebra course covering group theory (group actions, Sylow's theorems, results regarding the symmetric and alternating groups, direct and semidirect products, solvable and nilpotent groups, Jordan-Holder theorem) and module theory (usage of Zorn's lemma, chain conditions, simple modules, Jacobson radical of a ring, Wedderburn theory). \\ \emph{Textbook: }Isaacs' \emph{Algebra, a Graduate Course} (1-8, 10-14).}

\cventry{A}{MAS 4301}{Elementary Abstract Algebra I}{Spring 2020}{Brian Curtin}%
  {Groups (cyclic groups, subgroups, quotient groups, isomorphism theorems, finitely generated abelian groups), rings and ideals. \\ \emph{Textbook: }Fraleigh's \emph{A First Course in Abstract Algebra}.}



\subsection{Mathematics --- Analysis and Geometry}

\cventry{A+}{MAP 5345}{Applied Partial Differential Equations}{Spring 2021}{Razvan Teodorescu}%
  {Graduate applied partial differential equations course covering Laplace's equation, general elliptic equations, linear and nonlinear wave and heat equations. \\ \emph{Textbook: }Evans' \emph{Partial Differential Equations}.}

\cventry{A+}{MTG 4302}{Introduction to Topology}{Spring 2021}{Thomas Bieske}%
  {Point-set topology introduction covering basic topological space constructions (subspaces, product spaces, quotient spaces), connectedness, compactness, the separation axioms, and countability axioms. \\ \emph{Textbook: }Munkres' \emph{Topology}.}

\cventry{A+}{MTG 4254}{Differential Geometry}{Fall 2020}{Thomas Bieske}%
  {Differential geometry of curves and surfaces. Curvature, Serret-Frenet frame and formulae. Regular surfaces, the Gauss map, the shape operator, fundamental forms, scalar/principal/Gaussian/mean curvature(s), Gauss-Bonnet theorem. \\ \emph{Textbook: }O'Neill's \emph{Elementary Differential Geometry}.}

\cventry{A+}{MTG 4214}{Modern Geometry}{Spring 2020}{Thomas Bieske}%
  {Perusal through 20th and 21st century geometry research. Course began with recollection of general metric space properties and then covered topics such as geodesically accessible points, p-modulus of curve families, Hausdorff dimension, fractals, sub-Riemannian spaces, and hyperbolic spaces.  \\ \emph{Textbook: }None.}

\cventry{A+}{MAA 4211}{Intermediate Analysis I}{Spring 2020}{Boris Shekhtman}%
  {Ordered fields, real number system. General metric topology, compactness, sequences and series, continuity, connectness, intermediate and mean value theorems, uniform convergence, derivatives, and the Riemann integral. \\ \emph{Textbook: }Rudin's \emph{Principles of Mathematical Analysis}.}



\subsection{Other Mathematics}

\cventry{A+}{MAD 4203}{Introduction to Combinatorics}{Fall 2020}{Theodore Molla}%
  {Covered each element of the twelvefold way. Rook polynomials, ordinary and exponential generating functions, recurrence relations, the pigeonhole principle, principle of inclusion-exclusion, Burnside's lemma, and Polya counting.   \\ \emph{Textbook: }Brualdi's \emph{Introductory Combinatorics}.}

\cventry{A+}{MAS 3105}{Linear Algebra}{Fall 2019}{Brendan Nagle}%
  {Standard proof-based linear algebra course. Gaussian elimination, matrices, linear transformations, vector spaces, bases, determinants, eigenvalues, and diagonalization.\\ \emph{Textbook: }Bretscher's \emph{Linear Algebra with Applications}.}

\cventry{A+}{MGF 3301}{Bridge to Abstract Mathematics}{Fall 2019}{Brendan Nagle}%
  {Standard ``introduction to proofs'' course. Topics covered included propositional logic, elementary number theory, well-ordering, strong and weak induction, naive set theory, relations, partitions and equivalence relations, and Cantor's diagonalization argument. \\ \emph{Textbook: }Smith's \emph{A Transition to Advanced Mathematics}.}

\cventry{A+}{MAC 2313}{Calculus 3}{Fall 2019}{Vanaja Venkataraman}%
  {Vector arithmetic, partial derivatives, quadric surfaces, double and triple integration, line integration, Stokes' theorem, divergence theorem. \\ \emph{Textbook: }Stewart's \emph{Calculus}.}


%\section{Honors and Awards}
%\begin{itemize}
  %\item Obtained \$2000 grant from the UVA mathematics department.
  %\begin{itemize}
   % \item Supported summer research in operator algebras alongside Dr. Benjamin Hayes.
  %\end{itemize}
  %\item Mathematics PhD Qualifying Exams
  %\begin{itemize}
   % \item I took the Real and Complex analysis qualifying exams for upcoming 2nd year PhD students and successfully passed both.
  %\end{itemize}
  %\item College Science Scholar, University of Virginia
  %\begin{itemize}
   % \item Selected as one of 18 undergraduates for a research-oriented science-mentoring program.
  %\end{itemize}
 % \item Echols Scholar, University of Virginia
%  \begin{itemize}
%    \item Selected for Honors Program (5\% of incoming freshman), w%aives general education requirements.
%  \end{itemize}
%\end{itemize}




\end{document}
