\documentclass[11pt,letterpaper,sans]{moderncv}
\moderncvstyle{classic}
\moderncvcolor{purple}

% \usepackage[scale=0.80]{geometry}
\usepackage[scale=0.90]{geometry}

%\addtolength{\voffset}{-3.5em}
\addtolength{\voffset}{-1.0em}
\addtolength{\textheight}{7em}

%\quote{``Find what you love and let it kill you'' --- Charles Bukowski}
%\let\originalrecomputecvlengths\recomputecvlengths
%\renewcommand*{\recomputecvlengths}{%
%\originalrecomputecvlengths%
%\setlength{\quotewidth}{0.8\textwidth}}



\name{James}{Harbour}
\title{UVA Math \& CS}
\phone{+1~(205)~876~4085}
\email{james.h.harbour@gmail.com}
\homepage{jameshharbour.net}
\social{gtr8rh@virginia.edu}
\social[github]{kyrem1}
% \lfoot{\emph{Last updated \today}}

\sethintscolumnlength{1cm}

\definecolor{color1}{rgb}{0.50,0.33,0.80}% purple
\begin{document}
\makecvtitle % Print the CV title
\vspace*{-3.5em}

\vspace*{-1em}
\begin{center}
\textcolor{color1}{\textit{``Find what you love and let it kill you''} --- Charles Bukowski}
\end{center}
% \cventry{Grade}{CourseID}{CourseName}{Semester}{Professor}%
%    {Description \\ \emph{Textbook: }TextAuthor \emph{TextName}.}

%\begin{center}
%  %I am a mathematics and computer science double major at the University of Virginia with a comprehensive undergraduate and graduate-level mathematics background. My primary interests are in operator algerbas and noncommutative geometry.
%
%\vspace{1.0em}
%  Following my PhD, I plan to pivot towards the financial sector with a focus on quantitative finance and algorithmic trading.
%\end{center}


\section{Education}
\textbf{Class of 2025 at the University of Virginia}, Mathematics and Computer Science double major
\begin{itemize}
  \item Completed the undergraduate requirements for mathematics major during high school.
  \item Completed the UVA mathematics PhD curriculum as a freshman; passed PhD qualifying exams as a freshman.
  \item \textbf{Graduate level courses}: Machine Learning, Topoloigcal Modular Forms, Operator Algebras, Random Walks on Groups, Functional Analysis, Harmonic Analysis, Complex Analysis, Measure Theory, Differential Topology, Algebraic Topology I \& II, Algebraic Geometry, Algebra I \& II, Partial Differential Equations, 
\end{itemize}



\section{Mathematics Experience}

\cventry{2024}{Vanderbilt Research}{Visiting researcher under Dr.\,Jesse Peterson}{studying von Neumann equivalence and deformation/rigidity theory}{supported by the \textit{Harrison Award}}{}
\cventry{2024}{Purdue Research}{Visiting researcher under Dr.\,Thomas Sinclair}{studying tracial joint spectral measures, type III von Neumann algebras, and biexact groups}{supported by the \textit{Ingrassia Grant}}{}
\cventry{2023}{UChicago REU}{}{Full participant}{Studying $ s $-Perimeter and Nonlocal Potential Theory}{}
\cventry{2022}{UVA Research}{}{I took an intensive one-on-one reading and research course in operator algebras with Dr.\,Benjamin Hayes. Living expenses were covered by DMS-2000105.}{}{}


\section{Selected Talks}
\cventry{10/23}{$ \alpha $-stable Levy Processes and Fractional Laplacians}{Random Walks on Groups lecture}{}{}{}
\cventry{07/23}{Asymptotics of the Fractional $s$-Perimeter}{University of Chicago REU}{}{}{}

\cventry{04/23}{Maximal rigidity for $L^2$-cohomology of Groups and Beyond}{UVA Operator Algebras seminar}{}{}{} 
\cventry{11/22}{Index Rigidity for type-$ II_1 $ Subfactors}{UVA Operator Algebras seminar}{}{}{}

\section{Selected Travel}
\cventry{01/24}{Joint Math Meetings}{San Francisco}{speaker}{supported by AMS undergraduate travel grant}{}
\cventry{10/23}{East Coast Operator Algebras Symposium}{Purdue University}{attendee}{supported by NSF grant DMS-2321632}{}
\cventry{10/23}{Virginia Operator Theory and Complex Analysis Meeting (VOTCAM)}{Richmond University}{attendee}{}{}
\cventry{05/23}{Noncommutative Geometry and Operator Algebras (NCGOA) Spring Institute}{Vanderbilt University}{attendee}{}{}
\cventry{01/23}{Joint Math Meetings}{Boston}{attendee}{supported by NSF grant DMS-2035183}{}
\cventry{10/22}{East Coast Operator Algebras Symposium}{Michigan State University}{attendee}{supported by NSF grant DMS-2035183}{}
\cventry{06/22}{Thematic Program in p-adic L-functions and Eigenvarieties - Undergraduate Workshop}{University of Notre Dame}{participant}{supported by NSF grant DMS-1904501}{}
  
\section{Financial Experience}
\textbf{Quant Education Chair for UVA's Alternative Investment Fund}\quad An investment club managing a portfolio of $ \$60,000 $ AUM with both systematic and discretionary trading strategies. Rigorous selection process with multiple interviews and a $ 3\% $ acceptance rate. In my role as Quant Education Chair, I run a seminar on the mathematics behind quantitative finance.

\section{Programming Experience}
\textbf{Proficient In} Java, C, C++, Assembly, Python, \LaTeX, SageMath, Mathematica. I have also taken courses in cybersecurity, computer architectures, machine learning, and algorithmic economics.

%\section{Honors and Awards}
%\begin{itemize}
%  \item Various Grants/Funding/Travel (see my website for more info)
%  \begin{itemize}
%    \item \$5000 \textit{Harrison Award}, supporting summer research visits to Vanderbilt and Purdue.
%    \item \$2000 \textit{Ingrassia Grant}, supporting summer research visits to Vanderbilt and Purdue.
%    \item \$2000 Supporting summer research in operator algebras alongside Dr. Benjamin Hayes.
%    \item \$500 For travel to the Notre Dame workshop in Elliptic Curves and Modular Forms
%    \item \$1000 Supporting travel fees to attend the East Coast Operator Algebras Seminar. 
%    \item \$1500 Supporting travel fees to attend the Joint Mathematics Meetings
%  \end{itemize}
%  \item Mathematics PhD Qualifying Exams - Passed Real \& Complex Analysis Exams (as a freshman)

  % TODO

 % \item Echols Scholar, University of Virginia
 % \begin{itemize}
 %   \item Selected for Honors Program (5\% of incoming freshman), waives gen-ed requirements.
 % \end{itemize}
  
%\end{itemize}

\end{document}
